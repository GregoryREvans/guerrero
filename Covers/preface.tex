\documentclass[12pt]{article}
\usepackage{fontspec}
\usepackage[T1]{fontenc}
\usepackage[papersize={17in, 22in}, margin=0.5in]{geometry}
\usepackage{graphicx}
\usepackage{tikz}
\usepackage{xltxtra,fontspec,xunicode}
\usetikzlibrary{calc}
\defaultfontfeatures{Scale=MatchLowercase}
\newlength{\drop}
\pagenumbering{gobble}
\parindent=0pt
\setmainfont{Didot}
\usepackage{microtype}
\usepackage{graphicx}
\graphicspath{ {./images/} }
\usepackage{amsmath}\usepackage[absolute]{textpos}
\TPGrid[0.5in, 0.25in]{23}{24}
\begin{document}

\begin{center}
\huge FOREWORD
\end{center}

\begingroup
\begin{center}
\leftskip0.7in
\hspace{10mm} $GUERRERO$ is a Spanish word that means ``Warrior." This piece is intended, in part, to be an homage to the surname of composer Francisco Guerrero Mar\'in and to his work $Rhea$ for twelve saxophones, but the piece is primarily inspired by the figurative notions of reflection, refraction, and illumination. $GUERRERO$ begins with an $Invocation$ to its musical forerunner and to the muses of the epic mythology that led to the names of many stars, planets, and moons, before setting off on a journey of electric metamorphosis.
\rightskip\leftskip
\phantom{text} \hfill (G.R.E.)
\end{center}
\endgroup

\vspace*{1.5\baselineskip}

\begingroup
\begin{center}
\leftskip1.6in
\hspace{10mm} Sailing on an ocean of time and memory, the Warrior matches measured combinations with breath and \ae ther, neither hot nor cold, neither wet nor dry. Where once there was nothing is now the electromagnetic scintillation of particulate light, pulsing through the circulation where sight is now known. Whirlpools of stellar, parallactic aberration help mark the distance as the Warrior drifts away. The iron sea is a preface to a violent birth.
\rightskip\leftskip
\phantom{text} \hfill (G.R.E.)
\end{center}
\endgroup  
  
\vspace*{1\baselineskip}

\begin{center}
\huge PERFORMANCE NOTES
\end{center}

\begin{center}
Score is transposed.
\end{center}

\begin{center}
\pmb{Microtones}:
\end{center}

\begin{center}
\includegraphics[width=0.5\textwidth]{microtones.png}
\end{center}

\begin{center}
Accidentals apply only to the pitch which they immediately precede, but persist through ties. Microtones may be achieved either through the embouchure or fingerings.
\end{center}

\begin{center}
The symbol ``$\circ$" over a note represents a mostly airy tone-color that still retains some pitch.
\end{center}

\begin{center}
The symbol ``\o" \ represents a tone-color that is halfway between a normal playing technique and the $\circ$ technique.
\end{center}

\begin{center}
A ``$+$" over a note indicates a ``tongue slap" or ``tongue pizz." technique.
\end{center}

\vspace*{0.5\baselineskip}

\begin{center}
\huge INSTRUMENTATION
\end{center}
\begin{center}
1 Sopranino Saxophone
\end{center}
\begin{center}
3 Soprano Saxophones
\end{center}
\begin{center}
6 Alto Saxophones
\end{center}
\begin{center}
5 Tenor Saxophones
\end{center}
\begin{center}
3 Baritone Saxophones
\end{center}
\begin{center}
2 Bass Saxophones
\end{center}
\begin{center}
1 Contrabass Saxophone
\end{center}

\vspace*{1\baselineskip}

\begin{center}
c.5'30''
\end{center}

\begin{textblock}{23.7}(0, 23.2)
\begin{center}
\pmb{GUERRERO} was composed for $The \ Frost \ Saxophone \ Ensemble$ with special thanks to Joey Speranzo.
\end{center}
\end{textblock}

\end{document}